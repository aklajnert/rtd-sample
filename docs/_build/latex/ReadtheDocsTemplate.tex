%% Generated by Sphinx.
\def\sphinxdocclass{report}
\documentclass[letterpaper,10pt,english]{sphinxmanual}
\ifdefined\pdfpxdimen
   \let\sphinxpxdimen\pdfpxdimen\else\newdimen\sphinxpxdimen
\fi \sphinxpxdimen=.75bp\relax

\PassOptionsToPackage{warn}{textcomp}
\usepackage[utf8]{inputenc}
\ifdefined\DeclareUnicodeCharacter
% support both utf8 and utf8x syntaxes
\edef\sphinxdqmaybe{\ifdefined\DeclareUnicodeCharacterAsOptional\string"\fi}
  \DeclareUnicodeCharacter{\sphinxdqmaybe00A0}{\nobreakspace}
  \DeclareUnicodeCharacter{\sphinxdqmaybe2500}{\sphinxunichar{2500}}
  \DeclareUnicodeCharacter{\sphinxdqmaybe2502}{\sphinxunichar{2502}}
  \DeclareUnicodeCharacter{\sphinxdqmaybe2514}{\sphinxunichar{2514}}
  \DeclareUnicodeCharacter{\sphinxdqmaybe251C}{\sphinxunichar{251C}}
  \DeclareUnicodeCharacter{\sphinxdqmaybe2572}{\textbackslash}
\fi
\usepackage{cmap}
\usepackage[T1]{fontenc}
\usepackage{amsmath,amssymb,amstext}
\usepackage{babel}
\usepackage{times}
\usepackage[Sonny]{fncychap}
\ChNameVar{\Large\normalfont\sffamily}
\ChTitleVar{\Large\normalfont\sffamily}
\usepackage{sphinx}

\fvset{fontsize=\small}
\usepackage{geometry}

% Include hyperref last.
\usepackage{hyperref}
% Fix anchor placement for figures with captions.
\usepackage{hypcap}% it must be loaded after hyperref.
% Set up styles of URL: it should be placed after hyperref.
\urlstyle{same}

\addto\captionsenglish{\renewcommand{\figurename}{Fig.\@ }}
\makeatletter
\def\fnum@figure{\figurename\thefigure{}}
\makeatother
\addto\captionsenglish{\renewcommand{\tablename}{Table }}
\makeatletter
\def\fnum@table{\tablename\thetable{}}
\makeatother
\addto\captionsenglish{\renewcommand{\literalblockname}{Listing}}

\addto\captionsenglish{\renewcommand{\literalblockcontinuedname}{continued from previous page}}
\addto\captionsenglish{\renewcommand{\literalblockcontinuesname}{continues on next page}}
\addto\captionsenglish{\renewcommand{\sphinxnonalphabeticalgroupname}{Non-alphabetical}}
\addto\captionsenglish{\renewcommand{\sphinxsymbolsname}{Symbols}}
\addto\captionsenglish{\renewcommand{\sphinxnumbersname}{Numbers}}

\addto\extrasenglish{\def\pageautorefname{page}}

\setcounter{tocdepth}{1}



\title{Read the Docs Template Documentation}
\date{Apr 18, 2019}
\release{1.0}
\author{Read the Docs}
\newcommand{\sphinxlogo}{\vbox{}}
\renewcommand{\releasename}{Release}
\makeindex
\begin{document}

\ifdefined\shorthandoff
  \ifnum\catcode`\=\string=\active\shorthandoff{=}\fi
  \ifnum\catcode`\"=\active\shorthandoff{"}\fi
\fi

\pagestyle{empty}
\sphinxmaketitle
\pagestyle{plain}
\sphinxtableofcontents
\pagestyle{normal}
\phantomsection\label{\detokenize{index::doc}}


Contents:


\chapter{Authors}
\label{\detokenize{authors:authors}}\label{\detokenize{authors::doc}}\begin{itemize}
\item {} 
Eric (New contributor)

\item {} 
Anthony

\end{itemize}


\chapter{Installation}
\label{\detokenize{installation:installation}}\label{\detokenize{installation::doc}}
Install the package with pip:

\begin{sphinxVerbatim}[commandchars=\\\{\}]
\PYGZdl{} pip install read\PYGZhy{}the\PYGZhy{}docs\PYGZhy{}template
\end{sphinxVerbatim}


\chapter{Template}
\label{\detokenize{readme:template}}\label{\detokenize{readme::doc}}
\$project will solve your problem of where to start with documentation,
by providing a basic explanation of how to do it easily.

Look how easy it is to use:
\begin{quote}

import project
\# Get your stuff done
project.do\_stuff()
\end{quote}


\section{Features}
\label{\detokenize{readme:features}}\begin{itemize}
\item {} 
Be awesome

\item {} 
Make things faster

\end{itemize}


\section{Installation}
\label{\detokenize{readme:installation}}
Install \$project by running:
\begin{quote}

install project
\end{quote}


\section{Contribute}
\label{\detokenize{readme:contribute}}\begin{itemize}
\item {} 
Issue Tracker: github.com/\$project/\$project/issues

\item {} 
Source Code: github.com/\$project/\$project

\end{itemize}


\section{Support}
\label{\detokenize{readme:support}}
If you are having issues, please let us know.
We have a mailing list located at: \sphinxhref{mailto:project@google-groups.com}{project@google-groups.com}


\section{License}
\label{\detokenize{readme:license}}
The project is licensed under the BSD license.


\chapter{Usage}
\label{\detokenize{usage:usage}}\label{\detokenize{usage::doc}}
To use this template, simply update it:

\begin{sphinxVerbatim}[commandchars=\\\{\}]
\PYG{k+kn}{import} \PYG{n+nn}{read}\PYG{o}{\PYGZhy{}}\PYG{n}{the}\PYG{o}{\PYGZhy{}}\PYG{n}{docs}\PYG{o}{\PYGZhy{}}\PYG{n}{template}
\end{sphinxVerbatim}


\chapter{Indices and tables}
\label{\detokenize{index:indices-and-tables}}\begin{itemize}
\item {} 
\DUrole{xref,std,std-ref}{genindex}

\item {} 
\DUrole{xref,std,std-ref}{modindex}

\item {} 
\DUrole{xref,std,std-ref}{search}

\end{itemize}



\renewcommand{\indexname}{Index}
\printindex
\end{document}